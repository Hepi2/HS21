% Options for packages loaded elsewhere
\PassOptionsToPackage{unicode}{hyperref}
\PassOptionsToPackage{hyphens}{url}
%
\documentclass[
]{article}
\usepackage{lmodern}
\usepackage{amssymb,amsmath}
\usepackage{ifxetex,ifluatex}
\ifnum 0\ifxetex 1\fi\ifluatex 1\fi=0 % if pdftex
  \usepackage[T1]{fontenc}
  \usepackage[utf8]{inputenc}
  \usepackage{textcomp} % provide euro and other symbols
\else % if luatex or xetex
  \usepackage{unicode-math}
  \defaultfontfeatures{Scale=MatchLowercase}
  \defaultfontfeatures[\rmfamily]{Ligatures=TeX,Scale=1}
\fi
% Use upquote if available, for straight quotes in verbatim environments
\IfFileExists{upquote.sty}{\usepackage{upquote}}{}
\IfFileExists{microtype.sty}{% use microtype if available
  \usepackage[]{microtype}
  \UseMicrotypeSet[protrusion]{basicmath} % disable protrusion for tt fonts
}{}
\makeatletter
\@ifundefined{KOMAClassName}{% if non-KOMA class
  \IfFileExists{parskip.sty}{%
    \usepackage{parskip}
  }{% else
    \setlength{\parindent}{0pt}
    \setlength{\parskip}{6pt plus 2pt minus 1pt}}
}{% if KOMA class
  \KOMAoptions{parskip=half}}
\makeatother
\usepackage{xcolor}
\IfFileExists{xurl.sty}{\usepackage{xurl}}{} % add URL line breaks if available
\IfFileExists{bookmark.sty}{\usepackage{bookmark}}{\usepackage{hyperref}}
\hypersetup{
  hidelinks,
  pdfcreator={LaTeX via pandoc}}
\urlstyle{same} % disable monospaced font for URLs
\usepackage[margin=1in]{geometry}
\usepackage{color}
\usepackage{fancyvrb}
\newcommand{\VerbBar}{|}
\newcommand{\VERB}{\Verb[commandchars=\\\{\}]}
\DefineVerbatimEnvironment{Highlighting}{Verbatim}{commandchars=\\\{\}}
% Add ',fontsize=\small' for more characters per line
\usepackage{framed}
\definecolor{shadecolor}{RGB}{248,248,248}
\newenvironment{Shaded}{\begin{snugshade}}{\end{snugshade}}
\newcommand{\AlertTok}[1]{\textcolor[rgb]{0.94,0.16,0.16}{#1}}
\newcommand{\AnnotationTok}[1]{\textcolor[rgb]{0.56,0.35,0.01}{\textbf{\textit{#1}}}}
\newcommand{\AttributeTok}[1]{\textcolor[rgb]{0.77,0.63,0.00}{#1}}
\newcommand{\BaseNTok}[1]{\textcolor[rgb]{0.00,0.00,0.81}{#1}}
\newcommand{\BuiltInTok}[1]{#1}
\newcommand{\CharTok}[1]{\textcolor[rgb]{0.31,0.60,0.02}{#1}}
\newcommand{\CommentTok}[1]{\textcolor[rgb]{0.56,0.35,0.01}{\textit{#1}}}
\newcommand{\CommentVarTok}[1]{\textcolor[rgb]{0.56,0.35,0.01}{\textbf{\textit{#1}}}}
\newcommand{\ConstantTok}[1]{\textcolor[rgb]{0.00,0.00,0.00}{#1}}
\newcommand{\ControlFlowTok}[1]{\textcolor[rgb]{0.13,0.29,0.53}{\textbf{#1}}}
\newcommand{\DataTypeTok}[1]{\textcolor[rgb]{0.13,0.29,0.53}{#1}}
\newcommand{\DecValTok}[1]{\textcolor[rgb]{0.00,0.00,0.81}{#1}}
\newcommand{\DocumentationTok}[1]{\textcolor[rgb]{0.56,0.35,0.01}{\textbf{\textit{#1}}}}
\newcommand{\ErrorTok}[1]{\textcolor[rgb]{0.64,0.00,0.00}{\textbf{#1}}}
\newcommand{\ExtensionTok}[1]{#1}
\newcommand{\FloatTok}[1]{\textcolor[rgb]{0.00,0.00,0.81}{#1}}
\newcommand{\FunctionTok}[1]{\textcolor[rgb]{0.00,0.00,0.00}{#1}}
\newcommand{\ImportTok}[1]{#1}
\newcommand{\InformationTok}[1]{\textcolor[rgb]{0.56,0.35,0.01}{\textbf{\textit{#1}}}}
\newcommand{\KeywordTok}[1]{\textcolor[rgb]{0.13,0.29,0.53}{\textbf{#1}}}
\newcommand{\NormalTok}[1]{#1}
\newcommand{\OperatorTok}[1]{\textcolor[rgb]{0.81,0.36,0.00}{\textbf{#1}}}
\newcommand{\OtherTok}[1]{\textcolor[rgb]{0.56,0.35,0.01}{#1}}
\newcommand{\PreprocessorTok}[1]{\textcolor[rgb]{0.56,0.35,0.01}{\textit{#1}}}
\newcommand{\RegionMarkerTok}[1]{#1}
\newcommand{\SpecialCharTok}[1]{\textcolor[rgb]{0.00,0.00,0.00}{#1}}
\newcommand{\SpecialStringTok}[1]{\textcolor[rgb]{0.31,0.60,0.02}{#1}}
\newcommand{\StringTok}[1]{\textcolor[rgb]{0.31,0.60,0.02}{#1}}
\newcommand{\VariableTok}[1]{\textcolor[rgb]{0.00,0.00,0.00}{#1}}
\newcommand{\VerbatimStringTok}[1]{\textcolor[rgb]{0.31,0.60,0.02}{#1}}
\newcommand{\WarningTok}[1]{\textcolor[rgb]{0.56,0.35,0.01}{\textbf{\textit{#1}}}}
\usepackage{longtable,booktabs}
% Correct order of tables after \paragraph or \subparagraph
\usepackage{etoolbox}
\makeatletter
\patchcmd\longtable{\par}{\if@noskipsec\mbox{}\fi\par}{}{}
\makeatother
% Allow footnotes in longtable head/foot
\IfFileExists{footnotehyper.sty}{\usepackage{footnotehyper}}{\usepackage{footnote}}
\makesavenoteenv{longtable}
\usepackage{graphicx,grffile}
\makeatletter
\def\maxwidth{\ifdim\Gin@nat@width>\linewidth\linewidth\else\Gin@nat@width\fi}
\def\maxheight{\ifdim\Gin@nat@height>\textheight\textheight\else\Gin@nat@height\fi}
\makeatother
% Scale images if necessary, so that they will not overflow the page
% margins by default, and it is still possible to overwrite the defaults
% using explicit options in \includegraphics[width, height, ...]{}
\setkeys{Gin}{width=\maxwidth,height=\maxheight,keepaspectratio}
% Set default figure placement to htbp
\makeatletter
\def\fps@figure{htbp}
\makeatother
\setlength{\emergencystretch}{3em} % prevent overfull lines
\providecommand{\tightlist}{%
  \setlength{\itemsep}{0pt}\setlength{\parskip}{0pt}}
\setcounter{secnumdepth}{-\maxdimen} % remove section numbering

\author{}
\date{\vspace{-2.5em}}

\begin{document}

\hypertarget{musterloesung-aufgabe-5.2s-glmm}{%
\subsection{Musterloesung Aufgabe 5.2S:
GLMM}\label{musterloesung-aufgabe-5.2s-glmm}}

\begin{center}\rule{0.5\linewidth}{0.5pt}\end{center}

\begin{quote}
Lese-Empfehlung Kapitel 4.3.1 von
\href{https://christophm.github.io/interpretable-ml-book/extend-lm.html\#glm}{Christopher
Molnar}
\end{quote}

\begin{quote}
Interessierte
\href{https://rpsychologist.com/r-guide-longitudinal-lme-lmer}{hier}
oder \href{https://rpubs.com/kaz_yos/glmm1}{hier}
\end{quote}

\begin{quote}
Download \href{17_Statistik5/RFiles/solution_stat5.2.R}{R-Skript}
\end{quote}

\begin{quote}
Download \href{17_Statistik5/solution_stat5.2.pdf}{PDF}
\end{quote}

\begin{center}\rule{0.5\linewidth}{0.5pt}\end{center}

\textbf{kommentierter Lösungsweg}

\begin{Shaded}
\begin{Highlighting}[]
\NormalTok{df <-}\StringTok{ }\NormalTok{nova }\CommentTok{# kopiert originaler Datensatz}

\CommentTok{# Genereiert eine Dummyvariable: Fleisch 1, kein Fleisch 0}
\NormalTok{df }\OperatorTok
\StringTok{  }\CommentTok{# entfernt Personen die sich ein Buffet Teller gekauft}
\StringTok{  }\KeywordTok{filter}\NormalTok{(label_content }\OperatorTok{!=}\StringTok{ "Hot and Cold"}\NormalTok{) }\OperatorTok
\StringTok{  }\CommentTok{# ihr könnt keine Angabe vernachlässigen, sind (nur) 54 Personen}
\StringTok{  }\KeywordTok{filter}\NormalTok{(age_group }\OperatorTok{!=}\StringTok{ "keine Angaben"}\NormalTok{) }\OperatorTok\StringTok{ }
\StringTok{  }\KeywordTok{mutate}\NormalTok{(}\DataTypeTok{label_content =} \KeywordTok{str_replace_all}\NormalTok{(.}\OperatorTok{$}\NormalTok{label_content, }
                                         \KeywordTok{c}\NormalTok{(}\StringTok{"Fisch|Geflügel"),"}\NormalTok{Fleisch}\StringTok{")) %>%   }
\StringTok{  mutate(meat = if_else(.$label_content == "}\NormalTok{Fleisch}\StringTok{", 1, 0)) %>%}
\StringTok{  # setzt andere Reihenfolge für die Hochschulzugehörigkeit, nur für die Interpretation}
\StringTok{  # nützlich: neu Referenzkategorie Studierende (vorher Mitarbeitende)}
\StringTok{  mutate(member = factor(.$member, levels = c("}\NormalTok{Studierende}\StringTok{", "}\NormalTok{Mitarbeitende}\StringTok{")))  }


\StringTok{# wie viele NA's hat es dirn (uns interessiert v.a. die responsevariable: meat)}
\StringTok{sum(is.na(df$meat)) #Amelia::missmap(df_)}
\end{Highlighting}
\end{Shaded}

\begin{verbatim}
## [1] 0
\end{verbatim}

\begin{Shaded}
\begin{Highlighting}[]
\CommentTok{# sieht euch die Verteilung zwischen Fleisch und  kein Fleisch an, }
\CommentTok{# beide kategorien kommen nicht gleich häufig vor, aber nicht super tragisch}
\KeywordTok{prop.table}\NormalTok{(}\KeywordTok{table}\NormalTok{(df}\OperatorTok{$}\NormalTok{meat)) }\CommentTok{# gibt die prozente an}
\end{Highlighting}
\end{Shaded}

\begin{verbatim}
## 
##         0         1 
## 0.4782123 0.5217877
\end{verbatim}

\begin{Shaded}
\begin{Highlighting}[]
\KeywordTok{table}\NormalTok{(df}\OperatorTok{$}\NormalTok{meat) }\CommentTok{# gibt die absoluten werte an}
\end{Highlighting}
\end{Shaded}

\begin{verbatim}
## 
##    0    1 
## 8560 9340
\end{verbatim}

\begin{Shaded}
\begin{Highlighting}[]
\CommentTok{# definiert das logistische Modell mit ccrs als random intercept und }
\CommentTok{# wendet es auf den Datensatz an}

\KeywordTok{library}\NormalTok{(lme4)}
\CommentTok{#dauert ein paar sekunden}
\NormalTok{mod0 <-}\StringTok{ }\KeywordTok{glmer}\NormalTok{(meat }\OperatorTok{~}\StringTok{ }\NormalTok{gender }\OperatorTok{+}\StringTok{ }\NormalTok{member }\OperatorTok{+}\StringTok{ }\NormalTok{age_group }\OperatorTok{+}\StringTok{ }\NormalTok{(}\DecValTok{1}\OperatorTok{|}\NormalTok{ccrs),}
              \DataTypeTok{data =}\NormalTok{ df, }\KeywordTok{binomial}\NormalTok{(}\StringTok{"logit"}\NormalTok{)) }

\CommentTok{# lasst euch das Modell anzeigen: sieht so aus, als ob v.a. Geschlecht eine }
\CommentTok{# Rolle spielt}
\CommentTok{# wahrnmeldung kann vernachlässigt werden (aufgrund der unicode resp. }
\CommentTok{# umlaute in den variablen)}
\KeywordTok{summary}\NormalTok{(mod0) }
\end{Highlighting}
\end{Shaded}

\begin{verbatim}
## Generalized linear mixed model fit by maximum likelihood (Laplace
##   Approximation) [glmerMod]
##  Family: binomial  ( logit )
## Formula: meat ~ gender + member + age_group + (1 | ccrs)
##    Data: df
## 
##      AIC      BIC   logLik deviance df.resid 
##  22611.1  22665.6 -11298.5  22597.1    17893 
## 
## Scaled residuals: 
##     Min      1Q  Median      3Q     Max 
## -2.9879 -0.8253  0.4831  0.7618  3.2980 
## 
## Random effects:
##  Groups Name        Variance Std.Dev.
##  ccrs   (Intercept) 1.047    1.023   
## Number of obs: 17900, groups:  ccrs, 1427
## 
## Fixed effects:
##                            Estimate Std. Error z value Pr(>|z|)    
## (Intercept)                -0.61178    0.07146  -8.561  < 2e-16 ***
## genderM                     0.85769    0.07384  11.615  < 2e-16 ***
## memberMitarbeitende         0.14252    0.11044   1.290  0.19690    
## age_group26 bis 34-jährig -0.23126    0.08914  -2.594  0.00948 ** 
## age_group35 bis 49-jährig -0.07394    0.13687  -0.540  0.58905    
## age_group50 bis 64-jährig -0.01069    0.17538  -0.061  0.95138    
## ---
## Signif. codes:  0 '***' 0.001 '**' 0.01 '*' 0.05 '.' 0.1 ' ' 1
\end{verbatim}

\begin{verbatim}
## Warning in abbreviate(rn, minlength = 11): abbreviate mit nicht-ASCII Zeichen
## genutzt
\end{verbatim}

\begin{verbatim}
## 
## Correlation of Fixed Effects:
\end{verbatim}

\begin{verbatim}
## Warning in abbreviate(rn, minlength = 6): abbreviate mit nicht-ASCII Zeichen
## genutzt
\end{verbatim}

\begin{verbatim}
##             (Intr) gendrM mmbrMt a_26b3 a_35b4
## genderM     -0.658                            
## mmbrMtrbtnd -0.076  0.029                     
## ag_26b34-ä -0.411 -0.025 -0.353              
## ag_35b49-ä -0.281  0.041 -0.663  0.467       
## ag_50b64-ä -0.208  0.020 -0.598  0.393  0.517
\end{verbatim}

\begin{Shaded}
\begin{Highlighting}[]
\CommentTok{## erste Interpretation: Geschlecht (Mann) und Alter (junge Personen) scheinen den Fleischkonsum positiv zu beeinflussen + Hochschulzugehörigkeit spielt keien Rolle}
\CommentTok{# d.h. könnte man vernachlässigen. Ich lasse aus inhaltlichen Gründen aber im Modell drin}

\CommentTok{# Pseudo R^2}
\KeywordTok{library}\NormalTok{(MuMIn)}
\KeywordTok{r.squaredGLMM}\NormalTok{(mod0) }
\end{Highlighting}
\end{Shaded}

\begin{verbatim}
## Warning: 'r.squaredGLMM' now calculates a revised statistic. See the help page.
\end{verbatim}

\begin{verbatim}
## Warning: The null model is correct only if all variables used by the original
## model remain unchanged.
\end{verbatim}

\begin{verbatim}
##                    R2m       R2c
## theoretical 0.03760018 0.2699647
## delta       0.03240612 0.2326719
\end{verbatim}

\begin{Shaded}
\begin{Highlighting}[]
\CommentTok{# das marginale R^2 (r2m) gibt uns die erklärte Varianz der fixen Effekte: hier 4% (das ist sehr wenig)}
\CommentTok{# das conditionale R^2 (r2c) gibt uns die erklärte Varianz für das ganze Modell }
\CommentTok{# (mit fixen und variablen Effekten): hier 27% (ganz ok, aber auch nicht super mega)}
\CommentTok{# für weitere Informationen: https://rdrr.io/cran/MuMIn/man/r.squaredGLMM.html }

\CommentTok{# zusätzliche Informationen, welche für die Interpretation gut sein kann}
\CommentTok{# berechnet den Standardfehler (mehr infos: https://www.youtube.com/watch?v=r-txC-dpI-E oder hier: https://mgimond.github.io/Stats-in-R/CI.html)}
\CommentTok{# weitere info: https://stats.stackexchange.com/questions/26650/how-do-i-reference-a-regression-models-coefficients-standard-errors}
\NormalTok{se <-}\StringTok{ }\KeywordTok{sqrt}\NormalTok{(}\KeywordTok{diag}\NormalTok{(}\KeywordTok{vcov}\NormalTok{(mod0)))}

\CommentTok{# zeigt eine Tabelle der Schätzer mit 95% Konfidenzintervall (KI)}
\CommentTok{# => Faustregel: falls 0 im KI enthalten ist, dann ist der Unterschied statistisch NICHT signifikant}
\NormalTok{tab1 <-}\StringTok{ }\KeywordTok{cbind}\NormalTok{(}\DataTypeTok{Est =} \KeywordTok{fixef}\NormalTok{(mod0), }\DataTypeTok{LL =} \KeywordTok{fixef}\NormalTok{(mod0) }\OperatorTok{-}\StringTok{ }\FloatTok{1.96} \OperatorTok{*}\StringTok{ }\NormalTok{se,}
              \DataTypeTok{UL =} \KeywordTok{fixef}\NormalTok{(mod0) }\OperatorTok{+}\StringTok{ }\FloatTok{1.96} \OperatorTok{*}\StringTok{ }\NormalTok{se)}

\CommentTok{# erzeugt die Odds Ratios}
\NormalTok{tab2 <-}\StringTok{ }\KeywordTok{exp}\NormalTok{(tab1)}
\end{Highlighting}
\end{Shaded}

\begin{center}\rule{0.5\linewidth}{0.5pt}\end{center}

\textbf{Methoden}

Die Responsevariable ``Fleischkonsum'' ist eine binäre Variable. Demnach
wird eine multiple logistische Regression mit den Prädiktoren ``Alter
(Gruppen)'', ``Geschlecht'' und ``Hochschulzugehörigkeit'' gerechnet. Da
in den Daten gewisse Individuen mehrmals vorkommen, wird das Individuum
(Variable ccrs) als variabler Effekt in das Modell aufgenommen.

\begin{center}\rule{0.5\linewidth}{0.5pt}\end{center}

\textbf{Ergebnisse}

Das Geschlecht und das Alter nehmen einen signifikanten Einfluss auf den
Fleischkonsum (siehe Table 1): Männer kaufen signifikant häufiger ein
fleischhaltiges Gericht als Frauen; junge Personen (15 bis 25-jährig)
kaufen signifikant häufiger ein fleischhaltiges Gericht in der Mensa. Es
sieht so aus, als ob die Hochschulzugehörigkeit auf den ersten Blick
keinen Einfluss nimmt. Aber man müsste auch die Interaktion zwischen
Geschlecht und Hochschulzugehörigkeit berücksichtigen, um ein
abschliessendes Bild zu bekommen. Das kleine marginale Pseudo-R\^{}2
zeigt auf, dass es nicht das ``beste'' Modell ist. Insbesondere die
tiefe Varianzaufklärung für die randomisierte Variable (r2c; ccrs)
scheint mit (nur) 4\% sehr gering. Das sind Hinweise dafür, dass das
Modell ggf. noch weitere Ebenen haben könnte (z.B. Standort Mensa).

\begin{longtable}[]{@{}lrrr@{}}
\caption{Modellschätzer (Coefficients) mit dazugehörigem 95\%
Konfidenzintervall}\tabularnewline
\toprule
& Coefficients & Lower Limit (LL) & Uppewr Limit (UL)\tabularnewline
\midrule
\endfirsthead
\toprule
& Coefficients & Lower Limit (LL) & Uppewr Limit (UL)\tabularnewline
\midrule
\endhead
Intercept & -0.61 & -0.75 & -0.47\tabularnewline
Männer & 0.86 & 0.71 & 1.00\tabularnewline
Mitarbeitende & 0.14 & -0.07 & 0.36\tabularnewline
26 bis 34-jährig & -0.23 & -0.41 & -0.06\tabularnewline
35 bis 49-jährig & -0.07 & -0.34 & 0.19\tabularnewline
50 bis 64-jährig & -0.01 & -0.35 & 0.33\tabularnewline
\bottomrule
\end{longtable}

Die Chance, dass Männer ein fleischhaltiges Gericht kaufen ist 2.36mal
(+136\%) höher als bei Frauen (siehe Table 2). Die Chance, dass 26 bis
34-jährige Personen ein fleischhaltiges Gericht kaufen ist kleiner
(-21\%) als bei den 15 bis 25-jährigen Personen.

\begin{longtable}[]{@{}lrrr@{}}
\caption{Odds Ratio (OR) mit dazugehörigem 95\%
Konfidenzintervall}\tabularnewline
\toprule
& OR & Lower Limit (LL) & Uppewr Limit (UL)\tabularnewline
\midrule
\endfirsthead
\toprule
& OR & Lower Limit (LL) & Uppewr Limit (UL)\tabularnewline
\midrule
\endhead
Intercept & 0.54 & 0.47 & 0.62\tabularnewline
Männer & 2.36 & 2.04 & 2.72\tabularnewline
Mitarbeitende & 1.15 & 0.93 & 1.43\tabularnewline
26 bis 34-jährig & 0.79 & 0.67 & 0.95\tabularnewline
35 bis 49-jährig & 0.93 & 0.71 & 1.21\tabularnewline
50 bis 64-jährig & 0.99 & 0.70 & 1.40\tabularnewline
\bottomrule
\end{longtable}

\end{document}
